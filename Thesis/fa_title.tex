%% -!TEX root = AUTthesis.tex
% در این فایل، عنوان پایان‌نامه، مشخصات خود، متن تقدیمی‌، ستایش، سپاس‌گزاری و چکیده پایان‌نامه را به فارسی، وارد کنید.
% توجه داشته باشید که جدول حاوی مشخصات پروژه/پایان‌نامه/رساله و همچنین، مشخصات داخل آن، به طور خودکار، درج می‌شود.
%%%%%%%%%%%%%%%%%%%%%%%%%%%%%%%%%%%%
% دانشکده، آموزشکده و یا پژوهشکده  خود را وارد کنید
\faculty{دانشکده مهندسی برق}
% گرایش و گروه آموزشی خود را وارد کنید
\department{گرایش کنترل}
% عنوان پایان‌نامه را وارد کنید
\fatitle{طراحی و ساخت ربات چهارپا
	\\[.75 cm]}
% نام استاد(ان) راهنما را وارد کنید
\firstsupervisor{دکتر محمد اعظم خسروی}
%\secondsupervisor{استاد راهنمای دوم}
% نام استاد(دان) مشاور را وارد کنید. چنانچه استاد مشاور ندارید، دستور پایین را غیرفعال کنید.
%\firstadvisor{نام کامل استاد مشاور}
%\secondadvisor{استاد مشاور دوم}
% نام نویسنده را وارد کنید
\name{محمد }
% نام خانوادگی نویسنده را وارد کنید
\surname{برآبادی}
%%%%%%%%%%%%%%%%%%%%%%%%%%%%%%%%%%
\thesisdate{شهریور 1402}

% چکیده پایان‌نامه را وارد کنید
\fa-abstract{
امروزه ربات‌ها نقش به‌سزایی در زندگی انسان ها دارند. ربات‌های متحرک به دلیل نحوه تعامل‌شان با جهان، با سایر ماشین‌ها متفاوت هستند. آنها می‌توانند بر اساس اعمال خودشان تغییراتی در محیط اطراف خود ایجاد کنند و به دنیای اطراف خود پاسخ دهند. کاربردهای متنوع ربات ها در راستای ساده‌تر کردن زندگی انسان‌ها باعث شده مهندسین توجه ویژه‌ای به ربات‌ها داشته باشند.
\\
از میان تمامی ربات‌های متحرک، ربات‌های چهارپا یک نوع ربات پادار هستند که به دلیل توانایی آن‌ها برای اکتشاف در همه انواع زمین‌ها، مشابه انسان و حیوانات، نسبت به ربات‌های چرخدار برتری دارند. مزیت ربات‌های چهارپا در این است که هنگام ایستادن نیز حالت پایدار خود را حفظ نموده، اما در هنگام راه رفتن احتیاج به کنترل کردن دارند. 
\\
در این پایان‌نامه، ابتدا با ربات‌ها و انواع آنها با تاکید بر ربات‌های چهارپای ساخته شده و کاربردشان آشنا شده و در ادامه وارد بحث طراحی مدل و ساخت ربات با استفاده از نرم‌افزار سالیدورکس شده‌ایم. در راستای این طراحی، یک مدل اولیه ایجاد شده و پس از رفع عیوب ربات، تلاش بر آن شد که عملکرد آن بهبود یابد و یک نمونه نهایی اصلاح شده پرینت شد. سپس به مبحث سخت‌افزار و قطعات استفاده شده مانند موتور، میکروکنترلر و ماژول‌های مختلف در ربات پرداخته شده است. در پایان نیز الگوریتم‌های استفاده شده و برنامه‌نویسی روی میکروکنترلر ربات شرح داده شده‌اند. 
}


% کلمات کلیدی پایان‌نامه را وارد کنید
\keywords{ربات متحرک\LTRfootnote{Mobile Robot}، ربات چهارپا\LTRfootnote{Quadrupedal Robot}، ربات پادار\LTRfootnote{Legged Robot}، سالیدورکس\LTRfootnote{SolidWorks}، الگوریتم}



\AUTtitle
%%%%%%%%%%%%%%%%%%%%%%%%%%%%%%%%%%
\vspace*{7cm}
\thispagestyle{empty}
\begin{center}
	\includegraphics[height=5cm,width=12cm]{besm}
\end{center}